\documentclass[12pt]{article}

\usepackage[margin=1in]{geometry}
\usepackage{hyperref}
\usepackage{amsmath}
\usepackage{xcolor}
\usepackage{graphicx}
\usepackage{amssymb}

\newcommand{\R}{{\mathcal R}}
\newcommand{\F}{{\mathcal F}}
\newcommand{\V}{{\mathcal V}}
\usepackage{lineno}\renewcommand\thelinenumber{\color{gray}\arabic{linenumber}}

%%\parskip=10pt

\begin{document}

\linenumbers

\title{Mathematics 4MB3 Assignment 2}
\author{Jason Pekos, Zachary Levine}
\date{\today}
\maketitle

\section*{Question One}

\subsection*{Part a}
Let the total infectious period be the time between when an individual enters $I_1$ and leaves $I_n$. Assume that at time $t = 0$, we have $I_{1_0}$ infectives in the first serially linked infectious compartment, $I_1$. Then, if we prevent contact with any susceptible individuals, the differential equation for $I_1$ becomes
\begin{linenomath}
\begin{align*}
\frac{dI}{dt} &= n \gamma I_1
\end{align*}
\end{linenomath}

If we solve this differential equation using separation of variables, we obtain
\begin{linenomath}
\begin{align*}
I_1(t) = I_{1_0} e ^{-n\gamma t}
\end{align*}
\end{linenomath}

If at time $t$, $I_1(t) = I_{1_0} e^{-n\gamma t}$ individuals are in $I_1$, then after time $t$, the proportion of individuals in $I_1$ is reduced by a factor of $e^{-n\gamma t}$, so that the proportion of individuals who have an infectious period shorter than $t$ is $1-e^{-n\gamma t}$. Since this is the cumulative density function of the time an individual spends in $I_1$, the probability density function is the derivative of this function, or $n\gamma e^{-n\gamma t}$. The mean of this distribution is.

\begin{linenomath}
\begin{align*}
\int_{0}^{\infty} t n\gamma e^{-n\gamma t} dt = \frac{1}{\gamma n}
 \end{align*}
\end{linenomath}

So the mean time spent in $I_1$ is $\frac{1}{\gamma n}$ Since the removal rate for every infective compartment is the same, the average person should spend the same amount of time in each compartment. Thus, the total infectious period is $n \frac{1}{\gamma n} = \frac{1}{\gamma}$, which is unchanged from the standard SIR model with one infectious compartment.

\subsection*{Part b}

$\R_0$ captures the speed of infectious disease spread, which the linear chain trick does not change, when applied to a model. By expanding $I$ into $n$ sub compartments, but forcing people to move between the compartments at $n$ times the normal rate, the extra compartments do not change the speed of disease spread in the model. In addition, since the sum of the infectious compartments $\sum_1^n{I_i}$ is the same as $I$ in the standard SIR model, the rate of new infections is identical in both models, which means $\R_0$ should be too.

\end{document}

